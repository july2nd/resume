\documentclass[11pt,a4paper]{moderncv}

% moderncv themes
%\moderncvtheme[blue]{casual}                 % optional argument are 'blue' (default), 'orange', 'red', 'green', 'grey' and 'roman' (for roman fonts, instead of sans serif fonts)
\moderncvtheme[blue]{classic}                % idem
\usepackage{xunicode, xltxtra}
\XeTeXlinebreaklocale "zh"
\widowpenalty=10000

%\setmainfont[Mapping=tex-text]{文泉驿正黑}

% character encoding
%\usepackage[utf8]{inputenc}                   % replace by the encoding you are using
\usepackage{CJKutf8}
  
% adjust the page margins
\usepackage[scale=0.8]{geometry}
\recomputelengths                             % required when changes are made to page layout lengths
\setmainfont[Mapping=tex-text]{Hiragino Sans GB}
\setsansfont[Mapping=tex-text]{Hiragino Sans GB}
\CJKtilde

% personal data
\firstname{刘仰东}
\familyname{}
\title{}               % optional, remove the line if not wanted
\mobile{18911761014}                    % optional, remove the line if not wanted
\email{lydong519@gmail.com}                      % optional, remove the line if not wanted
%% \quote{\small{``Do what you fear, and the death of fear is certain.''\\-- Anthony Robbins}}
\nopagenumbers{}
\begin{document}
\maketitle
\section{工作经历}
\cventry{2016.12-至今}{IBM Machine Learning on z/OS Team}{}{}{}{主要负责应用容器化和产品持续构建.期间通过有效地采用多种开源工具,极大地提升了产品的开发,构建和测试效率,并获得了2017年 IBM Team Delivery Excellence Award.主要工作内容包括以下几个方面:\\
	1.使用Docker 容器化应用,并使用Kubernetes 对容器进行编排和管理.\\
	2.使用Jenkins 和Shell 脚本实现Docker Images 的自动化构建.\\
	3.使用Ansible 实现产品的自动化部署.\\
	4.使用Selenium 和Python 脚本实现产品的 Web UI 部分的自动化测试.\\
	5.负责Docker Registry,Jenkins Cluster,Nexus,Spark,Hadoop的搭建和维护.
}
\cventry{2016.04-2016.12}{IBM Commerce on Cloud Team}{}{}{}{主要负责数据库查询工具AutoQuery 的开发,Docker化,自动化构建和部署.\\
	1.使用Java,JSP,Jquery,Bootstrap,Liberty,DB2 完成AutoQuery 的核心功能.\\
	2.使用Docker实现AutoQuery 的容器化.\\
	3.使用Jenkins实现AutoQuery的自动化构建.\\
	4.使用IBM Urban Code Deploy(UCD)实现AutoQuery的自动化部署.\\
	5.使用ELK 实现AutoQuery 日志的收集和展示.\\
	6.使用Registor,Consul,Consul Template实现AutoQuery 的服务发现.\\
	7.实现了AutoQuery在Marathon/Mesos上的自动化部署.
}
\cventry{2015.07-2016.04}{IBM Commerce on Prem Team}{}{}{}{主要参与跨境电商平台项目 Cross Boarder Trade,用到的技术主要有Java、JQuery、Spring、JSP、DB2 等.\\
	1.参与商品Catalog模块的开发和UT.\\
	2.参与Cart 模块开发和UT.\\
	3.参与Calculation 模块的开发和UT.\\
	4.参与Passport模块的开发和UT.
}
\vspace*{0.1\baselineskip}

\section{实习经历}
\cventry{2015.04-2015.06}{IBM Commerce on Cloud Team}{}{}{}{主要参与OMNI Service项目,所做工作主要是开发购物车模块和UT. 主要用到的技术有Java、DB2等.}
\cventry{2014.06-2015.02}{滴滴出行 产品技术部大数据组}{}{}{}{方向为反作弊,主要工作为挖掘和分析作弊特征,制定反作弊策略.\\
	1.期间使用Redis 替换原有的HBase 查询机制,提升查询速度约200多倍,并使用Shell 脚本完成分布式部署.\\
	2.使用PHP,MySql 完成后台作弊订单审核模块开发.\\
	3.使用并查集建立乘客唯一性标识,并通过挖掘出租车轨迹特征,提高召回率16\%之多.
}
\vspace*{0.1\baselineskip}

\section{教育经历}
\cventry{2012.09-2015.07}{硕士}{中国科学院大学 计算机软件与理论专业(保送)}{}{}{}
\cventry{2008.09-2012.07}{本科}{山东大学 计算机科学与技术专业}{}{}{}  

%%\section{研究经历}
%%\vspace*{0.2\baselineskip}
%%\cventry{2013.03-2015.07}{中国互联网络信息中心(CNNIC) 技术发展研究部前沿技术研究组}{}{}{}{主要负责发改委专项——国家物联网标识管理公共服务平台(NIOT)注册系统的迁移维护,包括DNS解析节点的建设;并担任平台子项目“条码数据库建设”的负责人,技术点包括网页爬虫、网页解析、无粘连验证码识别破解、数据库建设.另外还有基于该平台的商品溯源编码标准及系统、供应链管理的研究.}
%%%%\vspace*{0.2\baselineskip}
%%%%\cventry{2011.09-2012.08}{山东大学威海计算机应用技术研究所 网络应用技术组}{}{}{}{负责威海莱迪康复医院网站的设计和开发,风机选型软件的开发.技术点包括SSH框架、JSP技术、MYSQL数据库等.另外,参与网络安全研究,包括网络抓包分析、网络协议分析、常用攻防技术,并产出毕业论文一篇,该论文被评为优秀毕业论文.}
%%
%%\section{项目经历}
%%\renewcommand{\baselinestretch}{1.2}
%%\vspace*{0.2\baselineskip}
%%\cventry{2014.04-2014.06}
%%{约影}
%%{}
%%{竞赛项目}{}
%%{一款基于LBS的社交应用.用户可以通过该APP发起约影活动,也可以查看周围人发起的活动参与约影.CS架构,客户端基于Android开发,服务器端使用SpringMVC框架+mysql数据库,两端通过Http、Json进行通信.}
%%
%%\vspace*{0.2\baselineskip}
%%\cventry{2013.09-2013.12}
%%{中国大数据技术创新与创业大赛}
%%{}
%%{竞赛项目}{}
%%{此比赛由中国科学院和中国计算机协会举办,我们的选题为“基于出租车GPS轨迹的位置服务”,赛题的任务是开发打车推荐算法,通过挖掘北京市出租车GPS历史数据生成推荐模型,然后根据用户的位置和当前时间,计算能打到车的概率及平均等待时间.我在该项目中主要负责数学建模、Hadoop平台使用、轨迹分析、核心代码的编写以及报告文档的编写.最终,我们队获得三等奖(前17/660支队伍).}


%%\vspace*{0.2\baselineskip}
%%\cventry{2012.11-2012.12}
%%{新浪微博心理预测分析}
%%{}
%%{课程设计项目}{}
%%{通过对新浪微博内容进行挖掘,分析博主的性格特点(内外向、抑郁程度等),内容主要包括数据标注、分类、文本挖掘、NLP、latex报告编写.}

%%
%%\renewcommand{\baselinestretch}{1.0}
%%
%%\section{发表论文}
%%\cventry{2015}
%%{\textbf{刘仰东}\textnormal{, 田野, 袁博, 毛伟}}
%%{一种基于车流量的司乘推荐模型}{科研信息化技术与应用}
%%{}{}{}
%%\cventry{2014}
%%{\textbf{Yangdong Liu}\textnormal{, Ye Tian, Bo Yuan, Chang Wu, WeiShuo Qian, Wei Mao}}
%%{Providing Useful Information for Passengers Based on TTF Model}{IEEE International Conference on Internet of Things 2014}
%%{}{}{}
%%
%%
%%\section{奖项荣誉}
%%\cventry{2015}{中国科学院计算机网络信息中心年终考核三等奖}{}{}{}{}
%%\cventry{2014}{中国科学院计算机网络信息中心年终考核二等奖}{}{}{}{}
%%\cventry{2013}{第一届中国大数据技术创新和创业大赛三等奖}{}{}{}{}
%%\cventry{2012}{山东大学优秀本科毕业生}{}{}{}{}
%%%%\cventry{2012}{山东大学优秀毕业论文}{}{}{}{}
%%%%\cventry{2011}{国家励志奖学金}{}{}{}{}
%%\cventry{2010, 2011}{山东大学校一等奖学金}{}{}{}{}
%%\cventry{2010, 2011}{山东大学校级优秀学生}{}{}{}{}
%%
%%
%%\section{个人技能}
%%\cventry{语言}{Java, Python, Shell, Ruby, C, C++, Php}{}{}{}{}
%%%%\cventry{系统}{Linux, Mac OS, Windows, Android}{}{}{}{}
%%\cventry{数据库}{DB2, Mysql, MongoDB, Redis}{}{}{}{}
%%\cventry{英语}{CET-4 588; CET-6 492 }{良好的英语读写能力}{}{}{}
%%
%%
%%%\section{自我评价}
%%%\cventry{经历}{具有较扎实的计算机相关基础知识和较丰富的项目开发经验}{}{}{}{}
%%%\cventry{团队}{具有良好的团队合作意识和团队合作经验}{}{}{}{}
%%%\cventry{能力}{对新技术感兴趣, 并具有较快的学习能力}{}{}{}{}
%%%\cventry{态度}{做人简单朴实, 工作认真负责}{}{}{}{}
%%
%%\section{其他信息}
%%%%\cventry{校园活动}{担任中国科学院计算机网络信息中心研究生会副主席和学术部部长}{}{}{}{}
%%\cventry{个人网站}{\url{http://www.lystar.com}}{}{}{}{}
%%\cventry{技术博客}{\url{http://blog.csdn.net/lydyangliu}}{}{}{}{}
%%\cventry{开源社区}{\url{https://github.com/lydgithub}}{}{}{}{}
%%\cventry{微信公众账号}{是你的bug}{}{}{}{}
%%\cventry{兴趣爱好}{羽毛球 乒乓球 吉他}{}{}{}{}


% \cvline{Photography}{\small Digital photography is my newest hobby.}

\closesection{}                   % needed to renewcommands
\renewcommand{\listitemsymbol}{-} % change the symbol for lists
\end{document}
%%\vspace*{0.2\baselineskip}
