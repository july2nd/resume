\documentclass[11pt,a4paper]{moderncv}

% moderncv themes
%\moderncvtheme[blue]{casual}                 % optional argument are 'blue' (default), 'orange', 'red', 'green', 'grey' and 'roman' (for roman fonts, instead of sans serif fonts)
\moderncvtheme[blue]{classic}                % idem
\usepackage{xunicode, xltxtra}
\XeTeXlinebreaklocale "zh"
\widowpenalty=10000

%\setmainfont[Mapping=tex-text]{文泉驿正黑}

% character encoding
%\usepackage[utf8]{inputenc}                   % replace by the encoding you are using
\usepackage{CJKutf8}
  
% adjust the page margins
\usepackage[scale=0.8]{geometry}
\recomputelengths                             % required when changes are made to page layout lengths
\setmainfont[Mapping=tex-text]{Hiragino Sans GB}
\setsansfont[Mapping=tex-text]{Hiragino Sans GB}
\CJKtilde

% personal data
\firstname{刘仰东}
\familyname{}
\title{}               % optional, remove the line if not wanted

\mobile{18911761014}                    % optional, remove the line if not wanted
\email{lydong519@gmail.com}                      % optional, remove the line if not wanted
%% \quote{\small{``Do what you fear, and the death of fear is certain.''\\-- Anthony Robbins}}

\nopagenumbers{}

\begin{document}

\maketitle

\section{教育经历}
\cventry{2012.09-2015.06}{硕士}{中国科学院 计算机软件与理论专业(推免)}{}{}{}
\cventry{2008.09-2012.06}{本科}{山东大学 计算机科学与技术专业}{}{}{}  

\section{实习经历}
\cventry{2014.06-至今}{滴滴打车 产品技术部大数据组}{}{}{}{方向为反作弊,主要工作为挖掘和分析作弊特征,制定反作弊策略。期间使用redis替换原有的hbase查询机制,提升查询速度约200多倍,并负责进行分布式部署。技术点设计php、shell、redis、hbase等。}
\cventry{2013.03-2014.06}{中国互联网络信息中心 技术发展研究部前沿技术研究组}{}{}{}{主要负责发改委专项——国家物联网标识管理公共服务平台(NIOT)注册系统的迁移维护,包括DNS解析节点的建设;并担任平台子项目“条码数据库建设”的负责人,技术点包括网页爬虫、网页解析、无粘连验证码识别破解、数据库建设。另外还有基于该平台的商品溯源编码标准及系统、供应链管理的研究。}

\section{项目经历}
\renewcommand{\baselinestretch}{1.2}


\vspace*{0.2\baselineskip}
\cventry{2014.05-2014.07}
{约影}
{}
{创业项目}{}
{一款基于地理位置的社交应用。用户可以通过该app发起约影活动,也可以查看周围人发起的活动参与约影。CS架构,客户端基于android开发,服务器端使用SpringMVC框架+mysql数据库,两端通过http、json进行通信。}

\vspace*{0.2\baselineskip}
\cventry{2013.09-2013.12}
{中国大数据技术创新与创业大赛}
{}
{竞赛项目}{}
{此比赛由中国科学院和中国计算机协会举办,我们的选题为“基于出租车GPS轨迹的位置服务”,赛题的任务是开发打车推荐算法,通过挖掘北京市出租车GPS历史数据生成推荐模型,然后根据用户的位置和当前时间,计算能打到车的概率及平均等待时间。我在该项目中主要负责数学建模、Hadoop平台学习、Google Map轨迹分析、部分核心代码的编写以及报告文档的编写。}


\vspace*{0.2\baselineskip}
\cventry{2012.11-2012.12}
{新浪微博性格分析器}
{}
{课程设计项目}{}
{通过对新浪微博内容进行挖掘,分析博主的性格特点(内外向、抑郁程度等),内容主要包括数据标注、分类、文本挖掘、NLP、latex报告编写。}


\vspace*{0.2\baselineskip}
\cventry{2011.09-2012.08}
{山东大学威海计算机应用技术研究所}
{}
{实验室项目}{}
{负责威海莱迪康复医院网站的设计和开发,风机选型软件的开发。技术点包括SSH框架、JSP技术、MYSQL数据库等。另外,参与网络安全研究,包括网络抓包分析、网络协议分析、常用攻防技术,并产出毕业论文一篇,该论文被评为优秀毕业论文。}


\renewcommand{\baselinestretch}{1.0}

\section{发表论文}
\cventry{2014}
{\textbf{Yangdong Liu}\textnormal{, Ye Tian, Bo Yuan, Chang Wu, WeiShuo Qian, Wei Mao}}
{Providing Useful Information for Passengers Based on TTF Model}{IEEE International Conference on Internet of Things 2014}
{}{}{}


\section{奖项荣誉}
\cventry{2014}{中国科学院计算机网络信息中心年终考核二等奖}{}{}{}{}
\cventry{2013}{第一届中国大数据技术创新和创业大赛三等奖}{}{}{}{}
\cventry{2012}{山东大学优秀本科毕业生}{}{}{}{}
\cventry{2010, 2011}{山东大学校一等奖学金}{}{}{}{}


\section{个人技能}
\cventry{语言}{ C = C++ = Java > Python = Shell = php}{}{}{}{}
\cventry{系统}{Linux, Mac OS , Android}{}{}{}{}
\cventry{英语}{CET-4 588; CET-6 492 }{良好的英语读写能力}{}{}{}


\section{自我评价}
\cventry{}{具有较扎实的计算机相关基础知识和较丰富的项目开发经验}{}{}{}{}
\cventry{}{具有良好的团队合作意识和团队合作经验}{}{}{}{}
\cventry{}{对新技术感兴趣,并具有较快的学习能力}{}{}{}{}
\cventry{}{做人简单朴实、工作认真负责}{}{}{}{}

\section{其他}
\cventry{校园活动}{担任中国科学院计算机网络信息中心研究生会副主席和学术部部长}{}{}{}{}
\cventry{个人网站}{\url{www.lydstar.com}}{个人博客}{}{}{}
\cventry{微信公众账号}{是你的bug}{}{}{}{}


% \cvline{Photography}{\small Digital photography is my newest hobby.}

\closesection{}                   % needed to renewcommands
\renewcommand{\listitemsymbol}{-} % change the symbol for lists

\end{document}
